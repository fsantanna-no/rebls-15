\documentclass{acm_proc_article-sp}

\usepackage{xspace}
\newcommand{\CEU}{\textsc{C\'{e}u}\xspace}
\newcommand{\code}[1] {{\small{\texttt{#1}}}}

\usepackage{color}
\definecolor{light}{gray}{0.87}
\definecolor{dark}{gray}{0.30}
%\definecolor{light}{rgb}{.90,.90,.90}
\definecolor{darkgreen}{rgb}{0,.50,0}
\definecolor{darkblue}{rgb}{0,0,.50}
\definecolor{darkred}{rgb}{.50,0,0}
\definecolor{darkpur}{rgb}{.50,0,.50}

\usepackage{listings}
%\usepackage{textcomp}
\lstset{
%columns=fullflexible,
%basicstyle=\ttfamily,
escapeinside={||},
mathescape=true,
    language=C, % choose the language of the code
    basicstyle=\fontfamily{pcr}\selectfont\scriptsize\color{black},
    keywordstyle=\color{black}\bfseries, % style for keywords
    numbers=none, % where to put the line-numbers
    numberstyle=\tiny, % the size of the fonts that are used for the line-numbers
    backgroundcolor=\color{light},
    showspaces=false, % show spaces adding particular underscores
    showstringspaces=false, % underline spaces within strings
    showtabs=false, % show tabs within strings adding particular underscores
    %frame=single, % adds a frame around the code
    tabsize=2, % sets default tabsize to 2 spaces
    %rulesepcolor=\color{gray}
    captionpos=b, % sets the caption-position to bottom
    breaklines=false, % sets automatic line breaking
    %breakatwhitespace=false,
    numbersep=2em,
    emph={par,or,hor,do,end,loop,await,emit,input,event,call,with,%
          var,and,then,else,C,return,pure,deterministic,nohold,finalize,%
          class, every, FOREVER, this, spawn, in, pool, watching, until, 
          interface, each, abort, when, signal, PROC, CHAN, SIGNAL, PAR, not,
          bool, data, tag, escape, new, traverse},
    emphstyle={\bfseries},
    commentstyle=\color{dark}\scriptsize,
    %xleftmargin=20pt,
    %xrightmargin=20pt,
    framesep=20pt,
    %upquote=true,
    %aboveskip={1.5\baselineskip},
}

\begin{document}

\title{Traverse}
%\subtitle{[Extended Abstract]

\numberofauthors{3}
\author{
\alignauthor
Francisco Sant'Anna \\
    \affaddr{Departamento de Inform\'atica --- PUC-Rio, Brazil} \\
    \email{fsantanna@inf.puc-rio.br}
\alignauthor
Hisham Muhammad \\
    \affaddr{Departamento de Inform\'atica --- PUC-Rio, Brazil} \\
    \email{hisham@inf.puc-rio.br}
\alignauthor
Johnicholas Hines \\
    \affaddr{Affiliation} \\
    \email{email@domain.com}
}

\maketitle
\begin{abstract}
We propose a structured mechanism to traverse recursive data structures 
incrementally.
\code{traverse} is ...
%a lexically scoped anonymous closure that can be invoked recursively.
%Each recursive instance may contain multiple lines of execution


MIX OF:
\begin{itemize}
    \item recursive calls to anonymous closures
    \item each instance---many co-routines
\end{itemize}

DESIGNED FOR \CEU:
\begin{itemize}
    \item lexical compositions
    \item static memory management
    \item bounded execution/memory
    \item reactive
    \item mutation
\end{itemize}


\end{abstract}

\category{D.3.3}{Programming Languages}{Language Constructs and Features}

\terms{Design, Languages}

\keywords{Incremental Computation, Structured Programming, Behavior Trees, Domain Specific Languages}

\section{Introduction}

...

... \CEU~\cite{ceu.sensys13,ceu.mod15}

...

\section{C\'eu}

\begin{itemize}
\item adts
\item description
\item expansion: pool / recursive spawn
\item mutation / safety / watching
\end{itemize}

\subsection{Data Types}

\CEU supports algebraic data types as an alternative to \code{struct}, 
\code{union}, and \code{enum} declarations of C.
For this reason, they differ from functional algebraic data types (\emph{a la} 
Haskell and ML) and are mutable and have static memory management.
Currently, there is no support for parametric types.

% rebls-15/list/list.ceu
\begin{figure}[t]
\begin{lstlisting}[numbers=left,xleftmargin=3em]
data List with
    tag NIL;
or
    tag CONS with
        var int   head;
        var List* tail;
    end
end
\end{lstlisting}
\caption{
A \code{List} data type with two constructors:
    \code{NIL} builds an empty list, and
    \code{CONS} builds an item \code{head} with a \code{tail} pointing to the 
rest of the list.
\label{lst.data.list}
}
\end{figure}

The \code{List} type of Figure~\ref{lst.data.list} is either an empty list 
\code{NIL} (line 2) or a \code{CONS} with a value in the \code{head} and a 
pointer to the rest of the list in the \code{tail} (lines 4--7).
The code in Figure~\ref{lst.data.mut} illustrates the use of the \code{List} 
data type.

% rebls-15/list/list.ceu
\begin{figure}[t]
\begin{lstlisting}[numbers=left,xleftmargin=3em]
do
    pool List[1] l1;
    l1 = new List.CONS(1,
              List.CONS(2,
               List.NIL()));
    _printf("%d, %d\n", l1:CONS.head,
                        l1:CONS.tail:NIL);
        // prints 1, 1
end
do
    pool List[] l2;
    l2 = new List.CONS(1,
              List.CONS(2,
               List.NIL()));
    l2:CONS.tail = new List.CONS(3, List.NIL());
    _printf("%d\n", l2:CONS.tail:CONS.head);
        // prints 3 (2 has been freed)
end
\end{lstlisting}
\caption{
Declaration, allocation, mutation, and deallocation of \code{List} data types.
\label{lst.data.mut}
}
\end{figure}

In the first block (lines 1--9), the \code{pool} declaration of \code{l1} 
represents the root of the list and also specifies a memory pool to hold all of 
its elements (line 1).
We limit \code{l1} to contain at most 1 element (i.e., \code{List[1]}).
The declaration also implicitly initializes the root to be the base case of the 
associated data type (i.e., \code{List.NIL}).
Then, we mutate the root element to point to a dynamically allocated list of 
two elements (lines 3--5).
The assignment infers the destination memory pool based on the \emph{l-val} of 
the expression (i.e., \code{l1}).
In this case, only the allocation of first element succeeds, with the failed 
allocation returning the base case of the data type (i.e., \code{List.NIL}).
The print command (line 6) outputs \emph{1, 1}: the \code{head} of the first 
element and the \code{NIL} check of the second element.
Due to static memory management, when \code{l1} goes out of scope, at the end 
of the block (line 9), all elements in the list are automatically deallocated.

In the second block (lines 10--18), we declare the \code{l2} pool with an
unbounded memory limit (i.e., \code{List[]} in line 11).
Now, the two-element allocation succeeds (lines 12--14).
Then, we mutate the tail of the first element to point to a newly allocated 
element, which also succeeds (line 15).
The print command (line 16) outputs \emph{3}, the new \code{head} of the second 
element.
In the moment of the mutation, the old subtree is completely removed from 
memory.
Finally, the end of the block (line 18) deallocates the pool along with all of 
its elements.

%TODO: invalid tag generates runtime error

Data types in \CEU has a number of limitations:
given that mutations deallocate whole subtrees, data types cannot represent 
general graphs (in particular, cannot represent cycles);
elements in different pools cannot be mixed;
and pointers to subtrees (i.e., weak references) must be observed as they can 
be deallocated at any time (discussed in Section~\ref{TODO}).

\subsection{Traverse}

...

\begin{figure}%[t]
\begin{lstlisting}[numbers=left,xleftmargin=3em]
pool List[] l;
l = new List.CONS(1,
         List.CONS(2,
          List.CONS(3,
           List.NIL())));

var int sum =
    traverse e in l do
        if e:NIL then
            escape 0;
        else
            var int sum_tail = traverse e:CONS.tail;
            escape sum_tail + e:CONS.head;
        end
    end;
_printf("sum = %d\n", sum);
\end{lstlisting}
%\rule{8.5cm}{0.37pt}
\caption{
Calculating the \emph{sum} of a list.
\label{lst.sum}
}
\end{figure}

The code in Figure~\ref{lst.sum} creates a list (lines 1--5) and traverses it 
to calculate the sum of elements (lines 7--16).

...

\section{Applications}

...

\subsection{Incremental Computation}

...

\begin{itemize}
\item gray binary generation?
\end{itemize}

...

\subsection{Behavior Trees}

...

\begin{itemize}
\item ?
\end{itemize}

...

\subsection{Domain Specific Languages}

...

\begin{itemize}
\item LOGO Turtle?
\end{itemize}

...

\begin{figure}%[t]
\begin{lstlisting}[numbers=left,xleftmargin=3em]
data Command with
    tag NOTHING;
or
    tag ROTATE with
        var int angle;
    end
or
    tag MOVE with
        var int pixels;
    end
or
    tag AWAIT with
        var int ms;
    end
or
    tag SEQUENCE with
        var Command* one;
        var Command* two;
    end
or
    tag REPEAT with
        var int      times;
        var Command* command;
    end
or
    tag PAROR with
        var Command* one;
        var Command* two;
    end
end
end
\end{lstlisting}
%\rule{8.5cm}{0.37pt}
\caption{ DSL for a LOGO turtle.
\label{lst.turtle.dsl}
}
\end{figure}

\begin{figure}%[t]
\begin{lstlisting}[numbers=left,xleftmargin=3em]
class Interpreter with
    pool Command[]& cmds;
    var  Turtle&    turtle;
do
    traverse cmd in cmds do
        watching cmd do
            if cmd:AWAIT then
                await (cmd:AWAIT.ms) ms;

            else/if cmd:ROTATE then
                do TurtleRotate with
                    this.turtle = turtle;
                    this.angle  = cmd:ROTATE.angle;
                end;

            else/if cmd:MOVE then
                do TurtleMove with
                    this.turtle  = turtle;
                    this.pixels = cmd:MOVE.pixels;
                end;

            else/if cmd:PAROR then
                par/or do
                    traverse cmd:PAROR.one;
                with
                    traverse cmd:PAROR.two;
                end

            else/if cmd:SEQUENCE then
                traverse cmd:SEQUENCE.one;
                traverse cmd:SEQUENCE.two;

            else/if cmd:REPEAT then
                loop i in cmd:REPEAT.times do
                    traverse cmd:REPEAT.command;
                end
            end
        end
    end
end
\end{lstlisting}
%\rule{8.5cm}{0.37pt}
\caption{ The turtle interpreter.
\label{lst.turtle.interpreter}
}
\end{figure}

\section{Related Work}

...

\section{Conclusion}

...

\bibliographystyle{abbrv}
\bibliography{rebls-15}
\balancecolumns
\end{document}

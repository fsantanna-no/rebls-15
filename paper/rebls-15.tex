\documentclass{acm_proc_article-sp}

\usepackage{xspace}
\newcommand{\CEU}{\textsc{C\'{e}u}\xspace}

\begin{document}

\title{Traverse}
%\subtitle{[Extended Abstract]

\numberofauthors{3}
\author{
\alignauthor
Francisco Sant'Anna \\
    \affaddr{Departamento de Inform\'atica --- PUC-Rio, Brazil} \\
    \email{fsantanna@inf.puc-rio.br}
\alignauthor
Hisham Muhammad \\
    \affaddr{Departamento de Inform\'atica --- PUC-Rio, Brazil} \\
    \email{hisham@inf.puc-rio.br}
\alignauthor
Johnicholas Hines \\
    \affaddr{Affiliation} \\
    \email{email@domain.com}
}
% There's nothing stopping you putting the seventh, eighth, etc.
% author on the opening page (as the 'third row') but we ask,
% for aesthetic reasons that you place these 'additional authors'
% in the \additional authors block, viz.
\additionalauthors{Additional authors: John Smith (The Th{\o}rv{\"a}ld Group,
email: {\texttt{jsmith@affiliation.org}}) and Julius P.~Kumquat
(The Kumquat Consortium, email: {\texttt{jpkumquat@consortium.net}}).}
\date{30 July 1999}
% Just remember to make sure that the TOTAL number of authors
% is the number that will appear on the first page PLUS the
% number that will appear in the \additionalauthors section.

\maketitle
\begin{abstract}
We propose a structured mechanism to traverse recursive data structures 
incrementally.
\texttt{traverse} is ...
%a lexically scoped anonymous closure that can be invoked recursively.
%Each recursive instance may contain multiple lines of execution


MIX OF:
\begin{itemize}
    \item recursive calls to anonymous closures
    \item each instance---many co-routines
\end{itemize}

DESIGNED FOR \CEU:
\begin{itemize}
    \item lexical compositions
    \item static memory management
    \item bounded execution/memory
    \item reactive
    \item mutation
\end{itemize}


\end{abstract}

\category{D.3.3}{Programming Languages}{Language Constructs and Features}

\terms{Design, Languages}

\keywords{Incremental Computation, Structured Programming, Behavior Trees, Domain Specific Languages}

\section{Introduction}

...

... \CEU~\cite{ceu.sensys13,ceu.mod15}

...

\section{Traverse}

...

\begin{itemize}
\item adts
\item description
\item expansion: pool / recursive spawn
\item mutation / safety / watching
\end{itemize}

...

\section{Applications}

...

\subsection{Incremental Computation}

...

\begin{itemize}
\item gray binary generation?
\end{itemize}

...

\subsection{Behavior Trees}

...

\begin{itemize}
\item ?
\end{itemize}

...

\subsection{Domain Specific Languages}

...

\begin{itemize}
\item LOGO Turtle?
\end{itemize}

...

\section{Related Work}

...

\section{Conclusion}

...

\bibliographystyle{abbrv}
\bibliography{rebls-15}
\balancecolumns
\end{document}
